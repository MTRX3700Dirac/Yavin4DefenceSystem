\documentclass[]{report}

\usepackage{graphicx}
\usepackage{bookmark}

% Title Page
\title{Technical Manuel \newline Yavin IV Defence System}
\author{Team Dirac}


\begin{document}
\maketitle

\chapter{Introduction}
\section{Document Identification}
This document describes the design of the Yavin IV Defence System. This document is prepared by Group DIRAC for assessment in MTRX3700 in 2014.

\section{System Overview}
<A brief statement of the purpose of the system or subsystem to which this document applies>
The Yavin IV Defence System is designed to provide accurate, low cost tracking for Death Stars and other similar objects.

\section{Document Overview}
<A short "road map" of the document, to provide an orientation for the reader. Summarise the purpose and contents of this document.>
This document describes the detailed technical specifications and functionality for the entire system. This includes the entire system design, implementation and usage.

\section{Reference Documents}
The present document is prepared on the basis of the following reference documents, and should be read in conjunction with them.
	<Insert relevant documents>
\subsection{Acronyms and Abbreviations}


\begin{center}
	\begin{tabular}{| l | c |}
		\hline
		Acronym & Meaning \\ \hline \hline
		Thing & Meaning of Thing \\ \hline
		Stuff & Meaning of Stuff \\
		\hline
	\end{tabular}
\end{center}

\chapter{System Description}
This section is intended to give a general overview of the basis for the Yavin IV Defence System system design, of its division into hardware and software modules, and of its development and implementation.

\section{Introduction}
The system is broken into 
<Give a technical description of the function of the whole system, in terms of its constituent parts, here termed modules. Generally, a module will have hardware and software parts.

\section{Operational Scenarios}
<Describe how the system is to be used. There may be several different ways that it ca be used perhaps involving different users, or classes of user. Present case diagrams here if you are using them. Each operational scenario is a part through a use case diagram - a way of using the system, with different outcomes or methods of use. You should also consider the various failures that may occur, and the consequences of these failures. >

\section{System Requirements}
The operational scenarios considered place certain requirements on the whole Yavin IV Defence system, and on the modules that comprise it.
<Statement of requirements that affect the system as a whole, and are not restricted to only a subset of its modules.>

\section{Module Design}
<Describe the breakdown of the design into functional modules. Each module probably contains both software and hardware.
Then include a section like the following 2.5 for each module. Not all of the sub-headings may be relevant for each module.>

The system was broken down into a number of independent modules which contain their own private variables, functions etc. Fig. \ref{fig:Modules} gives an approximate diagrammatic representation of the way the modules fit together.

\begin{figure}
\centering
\includegraphics[width=0.7\linewidth]{../Diagrams/Modules}
\caption[Modules]{Conceptual Diagram of the module breakdown and interaction between modules.}
\label{fig:Modules}
\end{figure}



\section{Serial}
\subsection{Description}
The serial module takes care of all communication (transmit and receive) over the serial UART (rs-232) port.

\subsection{Functional Requirements}
\subsubsection{Inputs}
The only external input to the system is characters or strings to transmit. \newline 
Either transmit, or transmitROM must be used depending on if the string is in RAM or ROM. The incorrect usage will result in printing nothing. \newline
Sending a string which in not null terminated will result in unexpected behaviour, most probably filling and overflowing the buffer until a null is reached. \newline
Sending a different datatype such as a float instead of a char will result in syntax errors, but other integer types may be casted into a char, truncating data and resulting in unexpected results. Again, the function only takes char's and integers must first be converted to a char array.

\subsubsection{Processes}
The following processes must be operational:
\begin{itemize}
	\item The circular buffer functionality - pushing and popping characters from buffers must work properly for the module to operate.
\end{itemize}


\subsubsection{Outputs}
The system must output the following to work correctly: 
\begin{itemize}
	\item exact characters received over the serial line in the correct order.
\end{itemize}

\subsubsection{Timing}
The serial module must be capable of:
\begin{itemize}
	\item Storing characters as soon as they are received over serial
	\item Retaining received characters until they are handled
	\item Retaining transmission characters until they can be transmitted
\end{itemize}

\subsubsection{Implemented Basic Functionality}
The serial module implements the following basic functionality:
\begin{itemize}
	\item Functioning receive and transmit serial interrupts
	\item Configure function to set up the module
	\item Separate circular buffers to store data received and data to be transmitted
	\item Public Function to add data to the transmission buffer
	\item Public Function to read from received buffer, and check if anything has been received
\end{itemize}

\subsection{Non-Functional Requirements}
\subsubsection{Performance}
The serial module should have the following performance characteristics:
\begin{itemize}
	\item Very fast ISR's - to affect background code, and other waiting interrupts as little as possible
	\item Very low ISR latency - So no characters are missed, and the the module transmits almost as soon as possible.
\end{itemize}

\subsubsection{Interfaces}
The following interface requirements are desirable:
\begin{itemize}
	\item Complete isolation/modularisation (e.g. no global interrupts) - the buffers are not accessible to the rest of the program
	\item Very simple, intuitive and appropriately named interface functions taking 1 or no arguments.
	\item As simple operation as possible - E.g. configureSerial() then transmit().
\end{itemize}

\subsubsection{Design Constraints}
The design of the serial module was constrained by the following:
\begin{itemize}
	\item Only High and Low ISR's on the PIC - Needed a public ISR function that is called when a serial interrupt is fired - Reduced modularity and interrupt response
	\item Very little memory on the PIC - buffers were restricted to 30 characters
\end{itemize}

\subsubsection{Implemented Additional Functionality}
In addition to the required functionality above, the serial module also offers the following functionality:
\begin{itemize}
	\item Push Null terminated strings to the transmit buffer
	\item Check if carriage return, or esc has been received
	\item Pop an entire string from the receive buffer up to a carriage return
	\item Clear the buffers
	\item Receiving backspace characters removes the last characters from the buffer (if not CR or ESC)
	\item Read a string form program memory and transmit
	\item Peek - Read character without removing from buffer
	\item Indicate if transmit buffer is empty (all messages sent)
\end{itemize}

\subsection{Conceptual Design:}
The serial module is INTERRUPT DRIVEN. This means that any background code can be running while the module is transmitting and/or receiving data over the serial line, and no serial data should ever be missed, overlooked or cut out. \newline
The module contains two circular buffers: a transmit buffer and a receive buffer. These buffers are NOT accessible by the rest of the program. Rather, the module provides a public function transmit() which takes a string, and places it into the transmit buffer. Anything in this buffer is then transmitted character by character when the transmit ready interrupt fires. \newline
Whenever a character is received over serial it is stored in the received buffer by an interrupt. Again this buffer is NOT accessible to the rest of the program. Rather it provides a number of functions to interact with it. The most commonly used of which is the readString() function, which returns everything in the buffer up to a carriage return (e.g. a line of input entered by the user). \newline
This serial module also allows users the opportunity to remove or change data they have already transmitted. If a backspace is received, then instead of storing it in the receive buffer it will remove the last received character from the buffer if that character is not a Carriage Return, Newline, or Escape operator. This enables a much more more user friendly system as otherwise there would be no way to fix any syntax error without pressing enter, getting an error and starting again. Furthermore, without this feature, if a user did backspace and change an input it would result in completely unexpected behaviour. \newline
Fig. \ref{fig:SerialModule} shows a conceptual diagram of the function of the serial module.

\begin{figure}
	\centering
	\includegraphics[width=0.7\linewidth]{"../Diagrams/Serial Module"}
	\caption[Serial Module]{Shows Conceptual Diagram of Serial Module -> Received input stored by interrupts into circular buffer to await pop commands. Transmit data pushed onto buffer which is transmitted via interrupts}
	\label{fig:SerialModule}
\end{figure}

\subsubsection{Assumptions Made}
The module assumes that the buffers will never be overfilled, i.e. is able to transmit data faster than being written into buffer, or that input is being handled in a timely fashion. Failing this, it is assumed that the oldest data (which is overwritten) is the least meaningful, and that loosing some data will not create catastrophic error in the system. It is recommended to include a wait if sending large blocks of text over serial.

\subsubsection{Constraints on Serial Performance}
The main constraints on the serial module performance are:
\begin{itemize}
	\item Baud Rate
	\item Interrupt Latency
\end{itemize}
The baud rate sets a maximum rate data can be transferred, which along with the buffer length restricts the rate at which data can be written to the transmit buffer without an overflow occurring.
The interrupt latency is the time delay between the interrupt firing and the actual event. This is generally very small (we found ~300$\mu$s), but a high latency could miss characters being received.

\subsection{Interface}
Refer to the Technical User Manual For detailed explanations of the interface functions and how to use them.

\section{Pan Tilt}
\subsubsection{Description}
The Pan Tilt module is responsible for interfacing and driving the pan tilt mechanism. This primarily consists of generating the PDM signals required to send dictate the position of the servo's.

\subsection{Functional Requirements}
\subsubsection{Inputs}
The module takes the following inputs
\begin{itemize}
	\item Direction Struct
\end{itemize}

\end{document}          
