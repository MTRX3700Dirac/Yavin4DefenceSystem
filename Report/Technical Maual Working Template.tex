\documentclass[10pt,a4paper]{report}
\usepackage[utf8]{inputenc}
\usepackage{amsmath}
\usepackage{amsfonts}
\usepackage{amssymb}
\author{Team Dirac}
\title{Mechatronics 3700 Death Star Tracker: Technical Manual}
\begin{document}

\maketitle

\begin{abstract}
	This document describes how each of the modules (including both software and hardware) work, and how to use them.
\end{abstract}

\part{Technical Documentation}
\chapter{Introduction}
\section{Document Identification}
This document describes the design and development of the "Yavin IV Orbital Tracking System". This is a proposed and prototyped design in response to the Rebel Alliance Commander Rye's request for a defence system to combat the imminent threat posed by The Empire, and their Death Star weapons platform. \newline This system is to effectively, efficiently and easily track a space-based planetary annihilator, approximately the size of a small moon. This document and design brief is prepared by Dirac Defence Limited for assessment in MTRX3700, year 2014. The was approved by lieutenants Reid and Bell, and small scale testing initiated. 

\section{System Overview}

\section{Document Overview}

\section{Reference Documents}

\subsection{Acronyms and Abbreviations}

\chapter{System Description}
\section{Introduction}

\section{Operational Scenarios}

\section{System Requirements}

\section{Module Design}

\section{Module Requirements: Module X}
\subsection{Functional Requirements}
\subsubsection{Inputs}
\subsubsection{Processes}
\subsubsection{Outputs}
\subsubsection{Timing}
\subsubsection{Failure Modes}

\subsection{Non-Functional (Quality of Service) Requirements}
\subsubsection{Performance}
\subsubsection{Interfaces}
\subsubsection{Design Constraints}


\section{Conceptual Design: Module X}
\subsection{Assumptions Made}
\subsection{Constraints on Module X Performance}

\chapter{User Interface Design}
\section{Classes of User}
\section{Interface Design: User Class Y}
\subsection{User Inputs and Outputs}
\subsection{Input Validation and Error Trapping}

\chapter{Hardware Design}
\section{Scope of X System Hardware}

\section{Hardware Design}
\subsection{Power Supply}
\subsection{Computer Design}
\subsection{Sensor Hardware}
\subsection{Actuator Hardware}
\subsection{Operator Input Hardware}
\subsection{Operator Output Hardware}
\subsection{Hardware Quality Assurance}

\section{Hardware Validation}

\section{Hardware Calibration Procedures}

\section{Hardware Maintenance and Adjustment}

\chapter{Software Design}
\section{Software Design Process}
\subsection{Software Development Environment}
\subsection{Software Implementation Stages and Test Plans}

\section{Software Quality Assurance}

\section{Software Design Description}
\subsection{Architecture}
\subsection{Software Interface}
\subsection{Software Components}

\section{Preconditions for Software}
\subsection{Preconditions for System Startup}
\subsection{Preconditions for System Shutdown}

\chapter{System Performance}
\section{Performance Testing}
\section{State of the System as Delivered}
\section{Future Improvements}

\chapter{Safety Implications}

\chapter{Conclusions}

\part{Appendicies}

\chapter{Supporting Calculations}

\chapter{DOxygen Documentation}

\chapter{Code Listing}

\end{document}