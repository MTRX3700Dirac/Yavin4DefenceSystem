\documentclass[]{report}

\usepackage{listings}
\usepackage{bookmark}
\usepackage{graphicx}
\usepackage{pdfpages}


% Title Page
\title{Interface Manual}
\author{Team Dirac}


\begin{document}
\maketitle

\begin{abstract}
	This document describes how each of the modules (including both software and hardware) work, and how to use them.
\end{abstract}

\part{Overview}
\chapter{Software Overview}
\section{Description}
This system is a state machine; there is a primary loop in main() which repeatedly calls functions based on the current state. Each of these functions can alter the state to transition based on their function.

\section{State Functions}
\begin{itemize}
	\item Initialisation
	\item Search - Tracking Module
	\item Track - Tracking Module
\end{itemize}

\part{Functionality}
\chapter{Factory Mode}
\section{Description}
The Yavin IV Death Star Tracker includes a Factory Mode not available to general users. This mode is used for in-factory calibration and testing.

\section{Entering/Exiting the Factory Mode}
The factory mode can be entered by inserting the correct key into the lock on the device, and turning the key. Similarly turning the key back and removing the key will exit the factory mode, and the system will return to the standard operational mode in remote serial mode.

\section{Functionality}
The Factory Mode adds the following functionality and commands to the standard Mode:
\begin{itemize}
	\item Calibrate the temperature sensor
	\item Calibrate the Azimuth servo motor
	\item Calibrate the Elevation servo motor
	\item Calibrate the range finding sensors
	\item View raw range data
	\item Set the ultrasound sample rate
	\item Set the infrared sample rate
	\item Set the number of ultrasound samples to use
	\item Set the number of infrared samples to use
	\item Show statics of sensor data
\end{itemize}


\part{Software}

\chapter{Interrupts}
\section{Description}
Many of the functions within different modules use interrupts, however there is only one service routine for high and low priority. Thus the ISR's and vectors have been moved into their own separate module, which call Module scope ISR functions.

\section{How It Works:}
For convenience interrupt macros have been defined in Common.h for each possible interrupt call. Querying one of these macros will return true if that interrupt fired an interrupt. These macros look like: TX\_INT, CCP1\_INT.\newline
Each module defines a macro (in its header file) which indicates which interrupts that module is using, by or-ing together the previously mentioned macros. When an interrupt fires it checks these macros for each module, and if true calls a 'module scope ISR'. These macros look like: SERIAL\_INT, RANGE\_INT etc. \newline
These module scope ISR's are just functions defined within each interrupt driven module, which are called whenever an interrupt associated with that macro is called. Thus these functions act as ISR's for the module without conflicting with anything else in the program.

\section{How To Use It:}
Follow the following steps to use an interrupt within a module.
\begin{enumerate}
	\item Make sure Interrupts.c is included in the build
	\item Define/Update the module macro in the module header file
	\item Check header file is included in Interrupts.c
	\item Check module macro is tested in ISR's (High and low ISR's)
	\item Use module ISR as general ISR
\end{enumerate}

\section{List of Public Functions}
This module has no public functions, or any real public interface as nothing within this module is called by anything within the program. Rather the list of included header files, and the macro checks must be updated when a new module is added.

\section{Example Code}
\subsection{Defining Module macro}
\begin{lstlisting}
//Serial Interrupt macro Defn. within Serial.h
#define SERIAL_INT (TX_INT || RC_INT)
\end{lstlisting}

\subsection{Updating Interrupt Includes}
\begin{lstlisting}
//Includes at the top of Interrupt.c
//Any module that uses interrupts will need to be included here:
#include "Tracking.h"
#include "Range.h"
#include "User_Interface.h"
#include "Serial.h"
#include "PanTilt.h"
#include "Temp.h"
#include "Menusystem.h"
//Add your module header file here...
\end{lstlisting}

\subsection{Updating ISR Check}
\begin{lstlisting}
if (SERIAL_INT)
{
serialISR();
}
if (PAN_TILT_ISR)
{
panTiltISR();
}
if (RANGE_INT)
{
rangeISR();
}
if (USER_INT)
{
userISR();
}
//Add additional checks here
\end{lstlisting}

\chapter{Serial}
\section{Description:}
The serial module takes care of all communication (transmit and receive) over the serial UART (rs-232) port.

\section{How It Works:}
The serial module is INTERRUPT DRIVEN. This means that any background code can be running while the module is transmitting and/or receiving data over the serial line. \newline
The module contains two circular buffers: a transmit buffer and a receive buffer. These buffers are NOT accessible by the rest of the program. Rather, the module provides a public function transmit() which takes a string, and places it into the transmit buffer. Anything in this buffer is then transmitted character by character when the transmit ready interrupt fires. \newline
Whenever a character is received over serial it is stored in the received buffer by an interrupt. Again this buffer is NOT accessible to the rest of the program. Rather it provides a number of functions to interact with it. The most commonly used of which is the readString() function, which returns everything in the buffer up to a carriage return (e.g. a line of input entered by the user). \newline
This serial module also allows users the opportunity to remove or change data they have already transmitted. If a backspace is received, then instead of storing it in the receive buffer it will remove the last received character from the buffer if that character is not a Carriage Return, Newline, or Escape operator. This enables a much more more user friendly system as otherwise there would be no way to fix any syntax error without pressing enter, getting an error and starting again. Furthermore, without this feature, if a user did backspace and change an input it would result in completely unexpected behaviour.

\section{How To Use It:}
\begin{itemize}
	\item Make sure Interrupts.c is included in the build
	\item Call the configureSerial() function to set up serial
	\item Call transmit() to begin transmitting something
	\item Use receiveEmpty() to check if anything has been received
	\item Use readString() to read everything received up to a carriage return
\end{itemize}

\section{List of Public Functions}
Below is a definitive list of all the public functions contained in this module, their use, their arguments and their returns.

\subsection{configureSerial}
\subsubsection{Usage:}
Used to set up the serial module for use. This function MUST be called before any serial functions will work. It need only be called ONCE at the beginning of the program.

\subsubsection{Arguments:}
This function takes no arguments

\subsubsection{Return Value:}
This function does not return anything.

\subsection{transmit}
\subsubsection{Usage:}
Used to transmit a string over serial UART. NOTE: The string MUST be NULL TERMINATED. The function will continue pushing data onto the transmit buffer until it hits a null terminator, which could fill the buffer and overwrite the given string if it is not properly null terminated. \newline
IMPORTANT: This function CANNOT be passed a literal. E.g. the following code will not work:
\begin{lstlisting}
transmit("Some Message");
\end{lstlisting} 
The literal is stored in ROM, and the pointer argument the function uses cannot access ROM memory. This will result in unexpected behaviour. Instead use the following code:
\begin{lstlisting}
char message[] = "Some Message";
transmit(message);
\end{lstlisting}
This loads the literal into RAM when the program begins.

\subsubsection{Arguments:}
A pointer to the start of a null-terminated string to begin transmitting.

\subsubsection{Return value:}
No return value

\subsection{transmitted}
\subsubsection{Usage:}
Used to determine if a transmit action has been completed. For example if coordination between serial transmissions and other actions is required then this function must be used.

\subsubsection{Arguments:}
This function takes no arguments

\subsubsection{Return Value:}
non-zero (true) if all transmit actions have been completed (i.e. the transmit buffer is empty).

\subsection{transChar}
\subsubsection{Usage}
Used to transmit a single character over serial. Works exactly the same way as transmit, and will thus not interfere with any other transmit function. Use of this function will remove the necessity of defining a string and null terminating it when only a single character has to be transmitted.

\subsubsection{Arguments:}
A character to transmit over serial

\subsubsection{Return Value:}
No return value

\subsection{receiveEmpty}
\subsubsection{Usage:}
Used to determine if anything has been received over serial.

\subsubsection{Arguments:}
No Arguments

\subsubsection{Return Value:}
Non-zero if the the receive buffer is empty (nothing new has been received).

\subsection{receivePeek}
\subsubsection{Usage:}
Used to inspect the next character received over serial without removing it from the buffer. (e.g. a subsequent receivePeek, or receivePop operation will return the same character).

\subsubsection{Arguments:}
No arguments

\subsubsection{Return Value:}
The next character received over serial

\subsection{receivePop}
\subsubsection{Usage:}
Used to read a character received over serial.

\subsubsection{Arguments:}
No Arguments

\subsubsection{Return Value:}
the next character received over serial

\subsection{receiveCR}
\subsubsection{Usage:}
Used to determine whether a Carriage Return has been detected over serial. Carriage Return is used as a "user input confirm" button.

\subsubsection{Arguments:}
No Arguments

\subsubsection{Return Value:}
Non-zero if a carriage return has been detected (i.e. there is a CR in the receive buffer)

\subsection{readString}
\subsubsection{Usage:}
Used to read an entire string received over serial.

\subsubsection{Arguments:}
Pointer to memory address to store received characters. \newline
NOTE: To avoid possible memory errors, Make sure to reserve the maximum number of characters that can be returned. The macro BUFFERLENGTH (the length of the receive buffer) is defined in CircularBuffers.h. E.g. the following code is advisable:
\begin{lstlisting}
#include "CircularBuffers.h"
...
char [BUFFERLENGTH] receivedData; 	//Reserve the max number of characters
readString(receivedData);		//Read string into address
\end{lstlisting}

\subsubsection{Return Value:}
No return value


\section{Example Code:}
\begin{lstlisting}
#include "Common.h"
#include "Serial.h"
#include "CircularBuffers.h"	//Only used for BUFFERLENGTH definition

void main()
{
	char message[] = "Serial Example";
	char received[BUFFERLENGTH];

	//Set up the Serial module for use
	configureSerial();
	
	//Transmit the message
	transmit(message);
	
	//Wait for a carriage return
	while(!receiveCR());
		
	//Read in returned string
	readString(received);
		
	//Transmit received string
	transmit(received);
	
	for (;;)
	{
		//Poll received data
		while (receiveEmpty());
		
		//Re-transmit received character
		transChar(receivePop());
	}
}
\end{lstlisting}

\chapter{PanTilt}
\section{Description:}
The Pan Tilt module is the module which controls the servo actuators which point the sensors in the correct direction. As such it generates the PDM's (Pulse Duration Modulated waves) necessary to run the servo's, and has public functions such as move, which directs the Pan Tilt mechanism to a certain direction.

\section{How It Works:}
This module uses a single output compare module to create the delays necessary to generate the PDM's of the desired duty cycle. Due to the interrupt latency of approximately 300$\mu$s, the PDM's are staggered so that the interrupt calls will never be closer than approximately 0.04 seconds. The full available duty cycle (1000$\mu$s) is divided over the given angular range. There is also an angular offset for calibration reasons. The Delay object is then created to define the necessary delays to create the desired PDM.
NOTE: This module is interrupt driven, so the Interrupt.c file must be included in the project for it to work.

\section{How To Use It:}
\begin{itemize}
	\item Ensure Interrupt.c is included in the build
	\item Include PanTilt.h in the file you want to use the module
	\item Call configureBase() function to set up the module
	\item Call Move() to send the base to an absolute direction
	\item Call increment() or incrementFine() to move the base to a relative direction (relative to its current)
	\item CalibratePanTilt() to calibrate the base direction
	\item RawDir() returns the calibrated direction
	\item getDir() returns the current direction
\end{itemize}

\section{List of Public Functions}
Below is a definitive list of all the public functions contained in this module, their use, their arguments and their returns.

\subsection{configureBase}
\subsubsection{Description:}
Configures the Pan Tilt module. After this function is called the module moves to its default position. Then the module is fully possibly moving to any desired direction with the move or the increment functions. This function need only be called once at the beginning of the program (or whenever you want to start using the PanTilt).

\subsubsection{Usage:}
This function is used to setup the Pan Tilt module so that one can direct the sensors to any direction desired. 

\subsubsection{Arguments:}
This function takes no arguments.

\subsubsection{Return Value:}
This function takes no arguments.

\subsection{Move}
\subsubsection{Description}
This function commands the pan tilt mechanism to the given position. For this function (0, 0) is defined as the midpoint in both the elevation and azimuth. In azimuth this means pointing along the length of the pan tilt mechanism, and in elevation this means pointing 45 degrees above the ground.

\subsubsection{Usage}
This function is used when you want to point at an absolute direction.

\subsubsection{Arguments}
This function takes a single object of type direction which contains the desired azimuth and elevation in degrees

\subsubsection{Return Value}
This function returns nothing.

\subsection{Increment}
\subsubsection{Description}
This function adds an offset onto the current pan tilt direction

\subsubsection{Usage}
This function is used when you want to move the direction pointed to by some amount, without particularly caring about the absolute direction.

\subsubsection{Arguments}
This function takes a single object of type Direction, which contains the incremental value in both the azimuth and elevation directions in degrees.

\subsubsection{Return Value}
This function returns nothing

\subsection{IncrementFine}
\subsubsection{Description:}
This function does the same thing as the Increment function, but it increments in the smallest possible resolution of the pan Tilt module, not degrees.

\subsubsection{Usage}
This function is used in lieu of increment when greater angular resolution is required, such as the tracking algorithm.

\subsubsection{Arguments}
This function takes a single object of type Direction (defined in common.h) which represents the desired relative direction. Unlike the increment function this argument is \emph{not in degrees}, but the smallest possible resolution, individual clock ticks on the high time. For the 10MHz clock on the minimal board this is expected to be ~0.04deg.

\subsubsection{Return Value}
None

\subsection{CalibratePanTilt}
\subsubsection{Description:}
This function calibrates the pan tilt offset to the given value. Calling this function ensures that future calls to the move function with the same direction passed as reference will return to this point. 

\subsubsection{Usage}
This function is used to change the reference position of the pan tilt mechanism.

\subsubsection{Arguments}
Takes a single argument in the form of a direction struct (defined in common.h). This direction corresponds to the reference direction to which the pan tilt mechanism is calibrated.

\subsubsection{Return Value}
This function does not return anything.

\subsection{CalibratePanTiltRange}
\subsubsection{Description:}
This function calibrates the pan tilt mechanism by changing the range of the mechanism. This function acts very similarly to the CalibratePanTilt function except that the reference position stays the same and the mechanism increments are changed so that any future move call to the same reference will return to this position.

\subsubsection{Usage}
This function is used to calibrate the pan tilt mechanism once the desired reference position has been set

\subsubsection{Arguments}
Takes a single argument in the form of a direction struct (defined in common.h). This direction corresponds to the reference direction to which the pan tilt mechanism is calibrated.

\subsubsection{Return Value}
This function does not return anything.

\subsection{SetMaxMin}
\subsubsection{Description}
This function sets the value of any accessible setting in the PanTilt module, such as the Maximum and Minimum Azimuth or inclination

\subsubsection{Usage}
This function is used to alter any accessible setting in the pan tilt module, primarily from user input

\subsubsection{Arguments}
The function takes the new value and an enumeration of the setting in which to write the value

\subsubsection{Return Value}
This function does not return anything

\subsection{GetMaxMin}
\subsubsection{Description}
This function returns the value in any accessible setting within the PanTilt module, such as the maximum and minimum azimuth and elevation

\subsubsection{Usage}
This function is used whenever a setting value needs to be accessed. Primarily for outputting to the user

\subsubsection{Arguments}
The setting (as an enumeration) to return

\subsubsection{Return Value}
The value currently in the queried setting (as a character).

\section{Example Code:}
\begin{lstlisting}
#include "Common.h"
#include "PanTilt.h"

void main()
{
	Direction dir;
	configureBase();	//Configure the base module

	dir.azimuth = 0;	//Define a direction to move to
	dir.inclination = 0;
	
	move(dir);			//Move the base to the defined direction
	
	dir.azimuth = 1;	//Define an increment
	dir.inclination = 1;
	
	increment(dir);		//Increment the current direction by 1 degree in the azimuth and the elevation
}
\end{lstlisting}

\chapter{Temp}
\section{Description}
The Temperature module controls the sampling and calibration of the temperature sensor.

\section{How It Works:}
The temperature module is very simply and mostly revolves around the AD sampling of the temperature sensor. The temperature is not expected to fluctuate much over a short period of time. For this reason the temperature is only sampled over reasonably long intervals (e.g. a minute). The temperature module simply stores the last measured temperature and returns this in the getTemp() function.

\section{How To Use It:}
\begin{itemize}
	\item Include Temp.h in your source code file
	\item Call configureTemp() to setup the module
	\item Call readTemp() or readTempx2() to re-sample the temperature
	\item Call getTemp() to return the last sample temperature
\end{itemize}

\section{List of Public Functions}
Below is a definitive list of all the public functions contained in this module, their use, their arguments and their returns.

\subsection{configureTemp}
\subsubsection{Description}
Configures the temperature module so that it can be used

\subsubsection{Usage:}
This function is used once at the beginning of the program to configure the temperature module for use.

\subsubsection{Arguments:}
None

\subsubsection{Return Value:}
None

\subsection{rawTemp}
\subsubsection{Description:}
This function returns the raw temperature, the sensor data without any calibration offset.

\subsubsection{Usage}
This function is primarily for when the user wants to display the raw temperature reading

\subsubsection{Arguments}
None

\subsubsection{Return Value}
The raw temperature reading in degrees

\subsection{readTemp}
\subsubsection{Description}
This function takes a sensor read of the temperature, returns the result and stores it in the temperature module

\subsubsection{Usage}
This function is called periodically to re-sample the temperature, to account for environmental differences during use.

\subsubsection{Arguments}
None

\subsubsection{Return Value}
The temperature read by the sensor (and calibrated).

\subsection{readTempx2}
\subsubsection{Description}
This function does the same thing as readTemp, but it returns twice the temperature in degrees, allowing for better resolution.

\subsubsection{Usage}
Used in lieu of readTemp when better resolution is required

\subsubsection{Arguments}
None

\subsubsection{Return Value}
Twice the temperature x2

\subsection{calibrateTemp}
\subsubsection{Description}
Calibrates the current sensor reading to a given value

\subsubsection{Usage}
Used to calibrate an offset in the temperature sensor module

\subsubsection{Arguments}
Takes the temperature to calibrate the temperature sensor to

\subsubsection{Return Value}
None

\subsection{getTemp}
\subsubsection{Description}
Returns the temperature from the last sample taken

\subsubsection{Usage}
Used to get the temperature without having to re-sample the temperature sensor.

\subsubsection{Arguments}
None

\subsubsection{Return Value}
The temperature from the last sample

\section{Example Code:}
\begin{lstlisting}
#include "Temp.h"

void main(void)
{
	unsigned char temperature;
	
	configureTemp();
	calibrateTemp(25); //Calibrate temperature sensor to 25deg
	
	temperature = readTemp();
	
	//Just keeps returning the same temperature
	for (;;) temperature = getTemp();
}
\end{lstlisting}

\chapter{Tracking}
\section{Description}
The tracking module contains the high level tracking and search algorithms used by the system. It then uses the Pan Tilt and Range modules to enact these modules.

\section{How It Works:}
The Tracking module contains two primary functions: search and track. Each of these functions is 'incremental' in nature. This means a single call will perform a single 'step', and these functions are called continuously to perform a constant action. \newline
The search algorithm is based on a simple raster like scan. The tracking algorithm tracks the centre of the target by finding the edges in both azimuth and elevation, and moving back to the centre. \newline
This module makes use of the panTilt, range and temp modules which in turn use interrupts. This means that Interrupts.c must be included it the build for the tracking module to work. A user need not worry about configuring the modules used by tracking, as everything is set up in the configureTracking function

\section{How To Use It:}
\begin{itemize}
	\item Include "Tracking.h"
	\item Call configureTracking()
	\item Repeatedly Call Search()
\end{itemize}

\section{List of Public Functions}
Below is a definitive list of all the public functions contained in this module, their use, their arguments and their returns.

\subsection{ConfigureTracking}
\subsubsection{Description:}
Configures the tracking module, and all subsequent modules used by the tracking module.

\subsubsection{Usage:}
This function must be called before making use of any functions within the tracking module.

\subsubsection{Arguments:}
None

\subsubsection{Return Value:}
None

\subsection{Search}
\subsubsection{Description:}
Performs a single increment of the Pan Tilt mechanism in a defined search pattern.

\section{Example Code:}


\chapter{User Interface}
\section{Description}
The user interface module is the equivalent of the serial module for the local user interface. It has a circular buffer and stores all the user inputs to the system to be handled later.

\section{How It Works:}
The module uses the external interrupt functionality of the PIC microcontroller to register when buttons are pressed by triggering an interrupt on a rising edge of the signal change. These interrupts cause a character for each button to be pushed into a circular buffer to be processed later. 

The confirm button press will add the new line character 0x0D to the buffer, whilst the back button will add the escape character 0x1B to the buffer.

The analogue to digital converter is used to read the value of the voltage drop across the output of potentiometer, and this is translated to an appropriate value for each menu.


\section{How To Use It:}
\begin{itemize}
	\item Include "User\_Interface.h"
	\item Call configUSER()
	\item Call the userEmpty() function to identify if input has been received.
	\item If so, call userPop() or userPeek()
\end{itemize}

\section{List of Public Functions}
Below is a definitive list of all the public functions contained in this module, their use, their arguments and their returns.

\subsection{configUSER}
\subsubsection{Description}
Configures the user interface for use.
\subsubsection{Usage:}
This function must be called during system start up so that the user interface buffers get initialised before any input is received. 

\subsubsection{Arguments:}
None
\subsubsection{Return Value:}
None

\subsection{userEmpty}
\subsubsection{Description}
Checks if the local interface receive buffer is empty.
\subsubsection{Usage:}
This function should be used before trying to call the userPop()  or userPeek() functions, as a check to see if there is any information to handle. 

\subsubsection{Arguments:}
None
\subsubsection{Return Value:}
A char with the value 1 if the receive buffer is empty, or 0 if the receive buffer contains at least one item.

\subsection{userPop}
\subsubsection{Description}
Removes the first item from the circular buffer and returns it.
\subsubsection{Usage:}
Call this function to retrieve the value of the first element in the buffer, and remove it from the buffer. Use this when processing the data.

\subsubsection{Arguments:}
None
\subsubsection{Return Value:}
The character representation of the button pressed.

\subsection{userPeek}
\subsubsection{Description}
Views the first item from the circular buffer and returns it, without removing it from the buffer.
\subsubsection{Usage:}
Call this function to retrieve the value of the first element in the buffer, and without removing it from the buffer.

\subsubsection{Arguments:}
None
\subsubsection{Return Value:}
The character representation of the button pressed.

\subsection{readDial}
\subsubsection{Description}
Read the position of the dial by splitting the value returned from the ADC into even discrete steps based on the given argument.
\subsubsection{Usage:}
Call this function with an argument equal to the number of states you would like to split the ADC result into. For instance, an input of 7 will return a value from 0 to 6 based on the value found from the potentiometer

\subsubsection{Arguments:}
An unsigned integer for the number of states to split the ADC result into. 
\subsubsection{Return Value:}
Which state number the ADC result was part of. 

\chapter{Range}
\section{Description}
The Rang module is responsible for sampling the range from the IR and ultrasonic sensors. It is also used to detect the object as the functions return zero if the object is out of range.

\section{How It Works:}

\section{How To Use It:}
\begin{itemize}
	\item 
\end{itemize}

\section{List of Public Functions}
Below is a definitive list of all the public functions contained in this module, their use, their arguments and their returns.

\subsection{}
\subsubsection{Usage:}

\subsubsection{Arguments:}

\subsubsection{Return Value:}

\section{Example Code:}

\chapter{Menu System}
\section{Description}
The Menu System Module is responsible for keeping track of the menu state, and parses all user inputs. The module interfaces with the serial and user interface modules to handle all inputs to the system.

\section{How It Works:}

\section{How To Use It:}
\begin{itemize}
	\item Include "MenuSystem.h"
	\item Call initialiseMenu(TrackingState) with a pointer to the tracking state. 
	\item Call serviceMenu() periodically to respond to user commands.
	\item Call dispTrack(TrackingData) to display a set of tracking data
	\item Call dispSearching() to display a searching message
	\item Call dispRawRange() to take a range measurement and print it on the screen
\end{itemize}

\section{List of Public Functions}
Below is a definitive list of all the public functions contained in this module, their use, their arguments and their returns.

\subsection{initialiseMenu}
\subsubsection{Description}
Initialises the menu objects and any modules that the menu requires. 
\subsubsection{Usage:}
Call this in before using any function from the menu system. 
\subsubsection{Arguments:}
A pointer to the global tracking state so the menu can change if the system is tracking or not.
\subsubsection{Return Value:}
None

\subsection{serviceMenu}
\subsubsection{Description}
Services the current menu by checking for local and serial input, and handling the input in the correct way based on the current menu.
\subsubsection{Usage:}
Call this periodically so the menu system can refresh and update. 
\subsubsection{Arguments:}
None
\subsubsection{Return Value:}
None

\subsection{dispTrack}
\subsubsection{Description}
Displays the tracking data of a given Tracking Data object by displaying the range, azimuth and elevation of the tracking hardware.
\subsubsection{Usage:}
Call this periodically when tracking to print current values. 
\subsubsection{Arguments:}
Tracking Data to print.
\subsubsection{Return Value:}
None

\subsection{dispSearching}
Clears the current line and prints a searching message
\subsubsection{Usage:}
Call this periodically when searching. 
\subsubsection{Arguments:}
None
\subsubsection{Return Value:}
None

\subsection{dispRawRange}
Requests data from the ultrasonic and IR sensors, clears the current line and prints the results.
\subsubsection{Usage:}
Call this to view the raw data, such as in the appropriate menu 
\subsubsection{Arguments:}
None
\subsubsection{Return Value:}
None

\chapter{LCD}
\section{Description}
The LCD module is responsible for interfacing with the LCD component and displaying LCD data in the local user mode.

\section{How It Works:}
The LCD module contains functions for interfacing to the specific LCD hardware model that the system is using.

\section{How To Use It:}
\begin{itemize}
	\item Include "LCD.h"
	\item Call configLCD()
	\item Call lcdWriteString(String, num) to write a specific string to a specific line.
	\item Call lcdWriteChar(char, line, column) to write a specific character to a specific place on the display.
\end{itemize}

\section{List of Public Functions}
Below is a definitive list of all the public functions contained in this module, their use, their arguments and their returns.

\subsection{configLCD}
\subsubsection{Usage:}
Call this to initialise the LCD hardware and software
\subsubsection{Arguments:}
None
\subsubsection{Return Value:}
None

\subsection{lcdWriteString}
\subsubsection{Usage:}
Call this to write a given null-terminated string to a specific line of the LCD.
\subsubsection{Arguments:}
The string to display, and which line (1 or 2) to display it on.
\subsubsection{Return Value:}
None

\subsection{lcdWriteChar}
\subsubsection{Usage:}
Call this function to write a specified character to a specific location on a line of the LCD.
\subsubsection{Arguments:}
The character to display, which line to display it on (1 or 2), and which column to display it to (1 to 16).
\subsubsection{Return Value:}
None

\part{Hardware}
\chapter{Assembly Instructions}
Follow the following instructions to assemble the system. (see the following sections for detailed information about each step)

\begin{enumerate}
	\item Connect the Power
	\item Connect the Servos
	\item Connect the ultrasonic
	\item Connect the IR sensor
	\item Connect the Temperature sensor
	\item Plug in Serial cable (if used)
\end{enumerate}

\section{Power Connections}
The system runs of a 5v regulated line powered by a 9v battery. The ultrasonic sensor has its own regulator to account for its power surges, so it is treated differently. All other components can be powered off the remaining 5v bus.

\begin{enumerate}
	\item Make sure battery is inserted, and has charge
	\item 
\end{enumerate}

\chapter{Pin Connections}
\subsection{Ultrasonic}
The ultrasonic sensor uses the following pins: 
\begin{itemize}
	\item INIT - RC3 (Ultrasound pin 4)
	\item ECHO - RC2 (Ultrasound pin 7)
	\item V+ - Ultrasonic 5v bus  (Ultrasound pin 9)
	\item GND - Ultrasonic 0v (Ultrasound pin 1)
	\item BINH - GND (Ultrasound pin 8)
	\item BLNK - GND (Ultrasound pin 2)
\end{itemize}
Refer to the diagrams \ref{PanTiltPins} below for the Ultrasonic and other pan tilt module pins.

\includepdf[pages={2}]{"../Datasheets/Servos and PanTilt/Notes_on_Pan-Tilt_Assembly"};
\label{PanTiltPins}

\subsection{IR}
The IR uses the following pins:
\begin{itemize}
	\item Output - AN0 (IR pin 1)
	\item Vcc - 5v bus (IR pin 3)
	\item GND - 0v bus (IR pin 2)
\end{itemize}

\subsection{Servos}
The servos make use of the following pins:
\begin{itemize}
	\item PDM (Az.) - RC0 (YEL)
	\item PDM (El.) - RC1 (YEL)
	\item V+ - 5v bus (RED)
	\item GND - 0v bus (BWN)
\end{itemize}

\subsection{Temperature Sensor}
The temperature sensor makes use of the following pins:
\begin{itemize}
	\item Output - AN1 
	\item V+ - 5v bus
	\item GND - 0v bus
\end{itemize}
Refer to the diagram below for the pin placements of the temperature sensor.

\includepdf[pages={2}]{"../Datasheets/Temp Sensor/TI_LM35"};
\label{TempSensorPins}


\subsection{LCD}
The LCD module uses the following pins
\begin{itemize}
	\item Data pins (0-7) - Port D (0-7)
	\item RS - RC4
	\item RW - RC5
	\item EN - RA4
	\item V+ - 5v bus
	\item GND - 0v bus
\end{itemize}

\subsection{Buttons (Interface)}
The buttons use the following pins:
\begin{itemize}
	\item Output button 1 - RB0
	\item Output button 2 - RB1 
	\item V+ - 5v bus
	\item GND - 0v bus (via pull down resistor)
\end{itemize}

\subsection{Potentiometer (Interface)}
The potentiometer uses the following pins:
\begin{itemize}
	\item Output - AN2 
	\item V+ - 5v bus
	\item GND - 0v bus
\end{itemize}

\subsection{Factory Switch}
The Factory switch uses the following pins:
\begin{itemize}
	\item Output - RB2
	\item V+ - 5v bus
	\item GND - 0v bus (via pull down resistor)
\end{itemize}

\chapter{Power Bus}
\subsection{Description}
The power bus is responsible for supplying power to the entire system. It must be capable of running off a single 9v battery, and supplying the regulated 5v required for all of the components used.

\subsection{Circuit Diagram}
Fig. \ref{fig:PowerBus} is a schematic of the power bus used in this system. The bypass capacitors are used to reduce current draw during sudden voltage switching. The ultrasonic sensor has a separate voltage regulator as even the 1mF bypass capacitor cannot completely regulate the 2A power spike when the ultrasonic sensor emits. By decoupling the regulators all other components aren't affected by current draw on the ultrasonic unless the internal battery voltage drop lowers the input of the regulator below 6v.
\begin{figure}
\centering
\includegraphics[width=1.2\linewidth]{"../Diagrams/Power Bus Rotated"}
\caption{Power Bus Circuit Diagram}
\label{fig:PowerBus}
\end{figure}

\part{Trouble Shooting}
\chapter{Hardware}
\section{Unexpected Target Address Programming PIC}
\subsection{Description}
While trying to program the PIC with a PICkit, or ICD a message just keeps coming up saying the target address is 0x00 or something, and cannot program the device.

\subsection{Cause}
This is a generic issue for just about any communications problem between the PIC and the programming device. David has mentioned several times that it can be caused by interference on the bus. Personally I have found it is usually the capacitor on MCLR to be the issue. MCLR is the external reset for the PIC, and is connected to the programmer so that the device can be reset to run the program etc. If the capacitor is too large the device cannot change the value fast enough, and you will just get zeros. If the cap is too small it may float somewhat and get random values.

\subsection{Solution}
The nominal cap value on MCLR (at least for the PICkit) is 0.1$\mu$F. If this is not what you have, perhaps try 0.1$\mu$F. We have also found success by removing the cap altogether. Otherwise see if you can shorten the cable, and/or check the programmer output on an oscilloscope to see if it is broken.

\section{'Sneezing' Motion}
\subsection{Description}
The servos seem to jump a little whenever the Ultrasonic sensor fires, giving the impression of 'sneezing'.

\subsection{Cause}
This problem is a result of connecting the servos and ultrasonic sensor to the same power bus. When the ultrasonic sensor res it draws 2A, which is enough to case a sizeable voltage drop on the power bus. This is enough to effect the position of the servos, and probably the IR sensor, although this is not as noticeable.

\subsection{Solution}
Solve this problem by connecting the ultrasonic sensor to the separate power bus provided.

\section{Jittery Servos}
\subsection{Description}
The servos are jittering around all over the place and everything in the software seems to be correct.

\subsection{Cause}
This may be caused by not supplying a common ground to the servos. The servos use a control PDM signal sent from the PIC. But the servos may be powered from a different power source and thus the voltage between the grounds can float.

\subsection{Solution}
Solve this by creating a common ground between the servos and the PIC - by connecting the grounds together.

\section{Servos Going Crazy}
\section{Description}
The pan Tilt module has been configured and used as documented, but the servos (or one of the servos) are going haywire.

\subsection{Cause}
Plugging the PDM outputs to the servos may be distorting the PDM signals. It is not entirely known why this is the case; check this on an oscilloscope with the PDM outputs both connected and disconnected to the servo inputs.

\subsection{Solution}
Try placing a voltage follower between the PDM output and the servo inputs. This acts as a buffer and attempts to prevent the servos input characteristics impacting the output signal from the PIC.

\chapter{Software}
\section{Module Just Not Working}
\subsection{Description}
A module, or functions within the module are just not working, and you have no idea why.

\subsection{Cause}
You may have forgotten to call the associated configure function, or left out the Interrupts.c file.

\subsection{Solution}
Make absolutely sure that the relevant configure function has been called. Also make sure that the Interrupts.c file has been added to the build (won't matter for non interrupt driven modules).

\section{Public Interface Functions Not Declared}
\subsection{Description}
Whenever you try to use the documented public interface functions the compiler tells you that they are not declared.

\subsection{Cause}
You may have forgotten to include the relevant header file, which contains external declaration of all the functions so that you can see/use them elsewhere in the program.

\subsection{Solution}
Include the relevant header file.

\section{Serial Not Transmitting}
\subsection{Description}
Serial has been set up as documented, but the transmit function is not working,
or works some of the time. May also be transmitting random stuff.

\subsection{Cause}
A likely source of issue is trying to pass a literal to the transmit function. As
detailed in the sections above, a literal is stored in program memory. The
pointer passed to the function can only access RAM.

\subsection{Solution}
To solve this problem initialise a string variable and pass a pointer to the variable. The variable is stored in RAM and initialised from the program memory literal.

\section{Serial Transmitting Garbage}
\subsection{Description}
All of the serial module steps have been followed, the serial is being transmitted
and received, but all data is garbage.

\subsection{Cause}
A likely source for this problem is an incorrect baud rate being set.

\subsection{Solution}
Make sure that MNML is declared in Common.h if using the minimal board,
or not set if using a picdem. Also check that the terminal is receiving at 9600
baud. If an external clock or oscillator is used that is neither 10MHz, nor 4MHz
then small changes must be made in the serial module.

\subsection{Cause}
Another possible cause is that the string you are passing is not correctly null
terminated. In this case you should expect to see some number (up to BUFFER-
LENGTH) characters transmitted which may or may not include the desired
string.

\end{document}          
