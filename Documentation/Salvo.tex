\documentclass[]{article}

%opening
\title{Using Salvo}
\author{Grant Louat}

\begin{document}

\maketitle

\begin{abstract}
This document describes how to set up salvo to run on an MPLAB X project.
\end{abstract}

\section{Installing Salvo:}
First download and install salvo from here: http://www.pumpkininc.com/content/get\_lite.htm. Simply enter some details and download the installation executable. \newline
Make sure you also download the file named salvo-lite-pic-3.2.3-c.zip.

\section{Before Setting up a Project:}
There is a compatibility issue with the c18 compiler we are using and salvo. This is resolved in the salvo-lite-pic-3.2.3-c.zip file. So go into ../salvo/lib/mcc18/ directory and copy all the files in the zip folder into this directory (replace the ones that are currently there).

\section{Setting up a Project:}
There are a couple of things you must do in the project before you can begin using salvo. First set up a new project (using c18 compiler), and the follow the following steps:
\begin{enumerate}
	\item Go to file->Project Properties. Then under c18->mcc18 and on the include directories tab click ..., and navigate to ../salvo/inc/. This tells the compiler where to find the salvo header files.
	\item Then in the project window (where you can see you source files) right click on libraries and click Add Library/Object file. Then in the window navigate to ../salvo/lib/mcc18/sfc18snt.lib.
\end{enumerate}
To use the salvo libraries

\end{document}
